\section*{Continuum Mechanics}
When studying the dynamics of a mediums with fluid or structure properties under the influence of forces, we need in some sense a good description of how these forces act and alter the system itself.

Any medium on a microscopic scale is built up of a structure of atoms, meaning we can observe "empty spaces " between each atom or discontinuities in the medium. Discribing any phsycial phenomen on larger scales in such a way are tedious and most often out of bounds due to the high number of particles. Instead we consider the medium to be continously distributed throughout the entire reagion it occupies. Hence we want to study some phsyical properties of the complete volume and not down on atomic scale. 

We consider the medium with continuum properties. By a continuum we mean a volume $V(t) \subset \mathbb{R}^3$ 
consiting of particles, which we observe for some properties. One property of interest could be the veloctity $\textbf{v}(x,t)$ for some point $x \in V(t)$ in time $t \in (0, T]$, which would mean the average velocity of the particles occupying this point \textit{x} at time \textit{t}  

\subsection*{Coordinate system, a matter of perspective}
We assume that our medium is continiously distributed throughout its own volume, and we start our observation of this medium
at som time $t_0$. As this choice is arbitary, we often choose to observe a medium in a stress free initial state. We call this state $V(t_0)$ of the medium as the \textit{reference configuration}. We let $V(t)$ for $t \geq t_0$ denote the \textit{current configuration}
As the medium is act upon by forces, one of the main properties of interest is the deformation. Let \^{x} be a particle in the reference cofiguration $\ha{x} ∈ \ha{V}$. 
We let x(\^x, t) be the new location of a particle \^x for time t such that $x \in V(t)$

\subsection*{Lagrangian}


\subsection*{Eulerian}
Considering a flow of fluid particles in a river, such a description is tedious 