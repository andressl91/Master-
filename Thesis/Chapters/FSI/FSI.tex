\chapter{Fluid Structure Interaction}


The consepts of Fluid-structure interaction are often introduced in several engineering feelds, for example biomechanics and hydrodynamics. As we will see throughout this chapter, one of the main challenges of this field that our governing equation describing fluid and solids are defined on different coordinate systems. Recall from chapter ?, that the solid equations are often described in the \textit{Lagrangian coordiante system}, while the fluid equations are on the contrary described in the \textit{Eularian coordinate system}. 

We define $\Omega$ in the \textit{reference configuration} be partitioned in a fluid domain $\hat{\Omega_f}$ and a structure domain $\hat{\Omega_s}$ such that
$\Omega = \hat{\Omega_f} \cup \hat{\Omega_s}$. Furhter we define the interface $\hat{\Gamma}$ as the intersection between these domains such that $\Gamma_i = \hat{\partial \Omega_f} \cap \hat{\partial \Omega_s}$.  A natural dilemma arises as the domain $\Omega$ undergoes deformation over time. If the natural coordinate system are used for $\hat{\Omega_f}$ and $\hat{\Omega_s}$, the domains doesn't match and the interface $\hat{\Gamma}$ doesn't have a general description for both domains. As such only one of the domains can be described in its natural coordinate system, while the other domain needs to be defined in some transformed coordinate system. \\ \\

The scope of FSI methods can formally be divided into \textit{interface-tracking} and \textit{interface-capturing } methods.\cite{Frei2016}. In the \textit{interface-tracking} litterature, the \textit{arbitary Lagrangian-Eulerian} formilation is dominant approach \cite{Richter2010a}, \cite{Frei2016}. In this approach the structure is given in its natural \textit{Lagrangian coordinate system}, while the fluid is transformed into an artificial \textit{Lagrangian} coordinate system. From this approach tracking of the interface $\hat{\Gamma}$ is more trivial, as it is fixed on a \textit{reference system} and can be tracked by mappings defined in Chapter 2. 

While \textit{interface-capturing } are also defined on a fixed mesh, the interface  $\hat{\Gamma}$ is now moving over the mesh element. Hence by its name, the inteface must be tracked as it is moving in time. 

Initially one of the \textit{interface-capturing } methods, the \textit{fully Eulerian apporach } was considered, but implementation of tracking the interface in time was proven unsuccessful. Therefore ALE approach was finally chosen for this thesis. As such both the \textit{fully Eulerian} and \textit{ALE} consepts, strengths and weaknesses will be introduced following sub-chapters. \\

\subsection{Fully Eulerian}
This method keeps the fluid in its \textit{Eulerian coordinates}, and such can be seen as the natural counterpart of the ALE method \cite{Wick2013}. First proposed by , \cite{Dunne2006}.

\subsection{Arbitary Lagrangian Eulerian}
MEANTION ALE CAN  BEHAVE EITHER EULERIAN AND LAGRANGIAN
The ALE method was initially developed to combine the strengths of the \textit{Lagranngian} and \textit{Eulerian} coordinate systems. As pointed out in chapter 2, the \textit{Lagrangian} description is useful for tracking particles as they are act upon by forces. Hence its main contribution is the ability to track interfaces and materials with history dependent properties.
In the ALE method one choose to keep the structure in its \textit{Lagrangian coordinate system}, while transforming the fluid domain into an artificial coordinate system similar to the \textit{Lagrangian coordinate system}. It is however important to note that there is no natural displacement in the fluid domain, hence this domain has no directly physical meaning \cite{Richter2010a}, \cite{Donea2004}. 
 
With this in mind, we will derive these transformations with the help of a new arbitary fixed reference system \ha{W}, following the ideas and approaches found in \cite{Richter2016}. Further we denote its deformation gradient as $\hat{F}_w$ and its determinant $\ha{J}_w$. Following the ideas from chapter 2, we introduce the invertibale mapping $\ha{T}_w : \ha{W} \rightarrow V(t)$ , with the scalar $\ha{f}(\ha{x}_W, t) = f(x,t) $ and vector $\hat{w}(\ha{x}_W, t) = \mathbf{w}(x,t) $ counterparts.\\ 
For $\ha{V} = \ha{W}$, $\ha{W}$ simply denotes the familiar Lagrangian description.
In the case $\ha{V} \neq \ha{W}$, $\ha{W}$ as pointed out earlier have no direct physical meaning.  Hence it is important to notice that the physical velocity $\hat{v}$ and the velocity of arbitary domain $\pder{\ha{W}_w}{t}$ doesn't neseccary coincide. This observation is essential, as we will soon see. \\

We will first define the transformation of spatial and temporal derivatives from $V(t)$ to $\ha{W}$ found in \cite{Richter2016}\\

\begin{lem}
Transformation of scalar spatial derivatives \\
\textit{Let f be a scalar function such that} $f: V(t) \rightarrow \mathbb{R}$, \textit{then} 
\begin{align}
\nabla f = \hat{F}_W^{-T} \hat{\nabla}\ha{f}
\end{align} 
\end{lem}

\begin{lem}
Transformation of vector spatial derivatives \\
\textit{Let \textbf{w} be a vector field such that} $\mathbf{w}: V(t) \rightarrow \mathbb{R}^d$, \textit{then} 
\begin{align}
\nabla \mathbf{w} = \hat{\nabla}\hat{w} \hat{F}_W^{-1} 
\end{align} 
\end{lem}

\begin{lem}
Transformation of scalar temporal derivatives \\
\textit{Let f be a scalar function such that} $f: V(t) \rightarrow \mathbb{R}$, \textit{then} 
\begin{align}
\pder{f}{t} = \pder{\ha{f} }{t} - (\hat{F}_W^{-1} \pder{\ha{T}_W}{t} \cdot \hat{\nabla}) \ha{f}
\end{align} 
\end{lem}

In addition we need a consistent way to transform the induced stresses in the \textit{Eulerian} coordinate system to $\ha{W}$. Hence we introduce the \textit{Piloa transformation}, found in most introductional courses in structure mechanics (ORANGE BOOK).
\\
\begin{lem}
T \\
\textit{Let \textbf{w} be a vector field such that} $\mathbf{w}: V(t) \rightarrow \mathbb{R}^d$, \textit{then the Piola transformation of w is defined by} 
\begin{align}
\mathbf{w} = \ha{J}_W \hat{F}^{-1}_W \hat{w}
\end{align} 
\end{lem}

The Piola transformation can be further extended to transform tensors, see \cite{Richter2016}, Orange book. This results is essential as it allows us to transform surface forces induced by the \textit{Cauchy stress tensor} on our arbitrary coordinate system $\ha{W}$. Lemma 1.4 brings us to \textit{the first Piola Kirchhoff stress tensor} $\hat{P} = \ha{J}_W \hat{\sigma}\hat{F}_W^{-T}$, mentioned in chapter 2. 

We now have the necessary tools to transform the conservation principles introduced in the fluid problem in chapter 2. Recall the Navier-Stokes equation defined in the \textit{Eulerian coordinate system} V(t).
\begin{align*}
&\rho \pder{\mathbf{v}}{t} + \rho \mathbf{v} \cdot \nabla \mathbf{v} =
\nabla \cdot \sigma + \rho \mathbf{f} \\
&\nabla \cdot \mathbf{v} = 0
\end{align*}
Using our newly introduced transformations of derivatives we map the equation to the arbitrary reference system $\ha{W}$. We will first consider the transformation of the\textit{material derivative}. By 
\begin{align*}
\frac{d \mathbf{v}}{dt}(x,t) = \pder{\mathbf{v}}{t}(x,t) + \nabla \mathbf{v}(x,t) \cdot \pder{x}{t} \\
\frac{d \mathbf{v}}{dt}(x,t) = \pder{\mathbf{v}}{t}(x,t) + \nabla \mathbf{v}(x,t) \cdot \mathbf{v}
\end{align*}
Note $\pder{x}{t}$ is the velocity of particles and not the transformation velocity $\pder{\ha{T}_W}{t}$. By lemma(1.1, 1.2, 1.3) we have  

\begin{align*}
\frac{d \mathbf{v}}{dt}(x,t) = 
\pder{\hat{v}}{t}(x,t) - (\hat{F}_W^{-1}\pder{\ha{T}_W}{t} \cdot \hat{\nabla})\hat{v}
+ \hat{F}_W^{-T}\hat{\nabla}\hat{v} \cdot \hat{v} \\
\mathbf{v} \cdot \nabla \mathbf{v} = \nabla \mathbf{v} \mathbf{v} = 
\hat{\nabla}\hat{v}\hat{F}_W^{-1}\hat{v} = (\hat{F}_W^{-1}\hat{v} \cdot \hat{\nabla})\hat{v} \hspace{4mm} \textit{FINN KILDE}
\end{align*}

These results can be used to show that

\begin{align*}
\pder{\mathbf{v}}{t} + \mathbf{v} \cdot \nabla \mathbf{v} =
\pder{\hat{v}}{t} + (\hat{F}_W^{-1}(\hat{v} - \pder{\ha{T}_W}{t}) \cdot \hat{\nabla}) \hat{v}
\end{align*}

By applying \textit{the first Piola Kirchhoff stress tensor} directly we transform the surface stress by 

\begin{align*}
\nabla \cdot \sigma = \nabla \cdot (\ha{J}_W \hat{\sigma}\hat{F}_W^{-T})
\end{align*}
In general, $\sigma$ is presumed on the form of a Newtonian fluid.
However special care must be taken, as $\sigma \neq \hat{\sigma}$ due to spatial derivatives within the tensor. Hence 
\begin{align*}
\sigma = -p I + \mu_f(\nabla \mathbf{v} + (\nabla \mathbf{v})^T \\
\hat{\sigma} = -\ha{p} I + \mu_f(\hat{\nabla}\hat{v}\hat{F}_W^{-1} +\hat{F}_W^{-T}\hat{\nabla}\hat{v}^T )
\end{align*} 

For the conservation of continuum we apply the \textit{Piola Transformation} such that

\begin{align*}
\nabla \cdot \mathbf{v} = \nabla \cdot (\ha{J} \hat{F}_W^{-1} \hat{v})
\end{align*}

\subsection*{Strong and weak coupling }
One of the major aspects of Fluid-structure interaction is the coupling of the fluid and solid equations.  As the total system is exerted by external forces, the interface $\hat{\Gamma}$ must fulfill the physical equilibrium of forces given by the two domains. Therefore, is is critical that the transmition of forces from the two domains are fulfilled in a consistent way.

