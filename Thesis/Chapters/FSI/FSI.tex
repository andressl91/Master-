\section*{Fluid Structure Interaction}
From the consepts of continuum mechanics we often expand our thoery by observing to mediums interacting with eachother as they
are act upon by forces. In this thesis we will look at how to mediums of fluid and structural properties interact. We will let our computational domain $\Omega$ in the \textit{reference configuration} be partitioned in a fluid domain $\hat{\Omega_f}$ and a 
structure domain $\hat{\Omega_s}$ such that
$\Omega = \hat{\Omega_f} \cup \hat{\Omega_s}$. Furhter we define the interface $\hat{\Gamma}$ as the intersection between these domains such that $\Gamma_i = \hat{\partial \Omega_f} \cap \hat{\partial \Omega_s}$ \newline \newline

\subsection{Fully Eulerian concept}
In contrast to the Lagrangian description of the structure, we no longer follow an individual partical $x(x_0, t)$ from its initial state. We keep our view-point of the structure domain $\Omega_s$ fixed, and observe as the continuum $\Omega_s(t)$ moves in time. This means that for some point $x \in \Omega_s(t)$ will be occupied by different particles $\hat{x}$ in time.

We will let $\textbf{v}_s$ denote the solid velocity and $\textbf{u}_s$ the solid displacement in the Eulerian formulation of the structure. We define the mapping $\hat{x} = T_s(x,t) = x - \textbf{u}(x,t)$ of an Eulerian coordinate of particle $x \in \Omega(t)_s$ back to its coordinate in the \textit{reference configuration} $x \in \Omega(t_0)_s$. \newline \newline INSERT FIGURE \newline 

Let $\textbf{F} = \nabla T = I - \nabla \textbf{u}$ be defined as the displacement gradient and further let 
$J = \text{det}\textbf{F}$, be its determinant.

