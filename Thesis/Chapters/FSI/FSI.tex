\chapter{Fluid Structure Interaction}


\section{Choice of coordinate system}
The concepts of Fluid-structure interaction are often introduced in several engineering fields, for example bio mechanics and hydrodynamics. As we will see throughout this chapter, one of the main challenges of this field that our governing equation describing fluid and solids are defined on different coordinate systems. 

We define $\Omega$ in the \textit{reference configuration} be partitioned in a fluid domain $\hat{\Omega_f}$ and a structure domain $\hat{\Omega_s}$ such that
$\Omega = \hat{\Omega_f} \cup \hat{\Omega_s}$. Further we define the interface $\hat{\Gamma}$ as the intersection between these domains such that $\Gamma_i = \hat{\partial \Omega_f} \cap \hat{\partial \Omega_s}$. 
The fluid-structure interaction problem is then defined by the fluid and solid equations, and the transmission of the \textit{kinematic} and \textit{dynamic} conditions on the interface $\hat{\Gamma}$. 

\begin{align}
\mathbf{v}_f = \mathbf{v}_s \\
\mathbf{\sigma}_f \cdot \mathbf{n} = \mathbf{\sigma}_s \cdot \mathbf{n}
\end{align}

 A natural dilemma arises as the domain $\Omega$ undergoes deformation over time. Recall from chapter ?, that the solid equations are often described in the \textit{Lagrangian coordinate system}, while the fluid equations are on the contrary described in the \textit{Eularian coordinate system}. If the natural coordinate system are used for $\hat{\Omega_f}$ and $\hat{\Omega_s}$, the domains doesn't match and the interface $\hat{\Gamma}$ doesn't have a general description for both domains. As such only one of the domains can be described in its natural coordinate system, while the other domain needs to be defined in some transformed coordinate system. \\ \\

Fluid-structure interaction problems are formally divided into the \textit{monolithic} and \textit{partitioned} frameworks.  In the monolithic framework all equations and interfaceconditions are solved simultaneously. If the \textit{kinematic} (1.1) and \textit{dynamic}(1.2) conditions are satisfied exactly at each timestep, the method denoted as a \textit{strongly coupled} which is typically for a monolithic approach. However this strong coupling yields contributes to a stronger nonlinear behaviour of the whole system \cite{Wick}. There are several nonlinear solution strategies for the monolithic approach..
Further the monolithic framework provides less modular software since the implementation often \textit{ad hoc} for a certain monolithic approach. 

In the \textit{partitioned} framework one solves the equations of fluid and structure subsequently. The strength of such an approach is the wast range of optimized solvers developed for both the fluid and solid. However one major problem is that the interface conditions are not naturally met, and as such certain sub-iterations is required to achieve these. 

Independent of framework, one of the major aspects of Fluid-structure interaction is the coupling of the fluid and solid equations through (1.1) and (1.2) .  As the total system is exerted by external forces, the interface $\hat{\Gamma}$ must fulfill the physical equilibrium of forces given by the two domains. Therefore, it is critical that the transmission of forces from the two domains are fulfilled in a consistent way. However as the interface conditions are not naturally met such as in a \textit{partitioned} approach, we one must distinguish between \textit{strongly} and \textit{weakly} coupled schemes. \textit{Partitioned} solvers often enforce a weakly coupled schemes, meaning (1.1) and (1.2) are not strictly enforced in the calculation. Such an approach is often sufficient for some fields such as computations within aeroelasticity \cite{Fernandez2009}. However by sub-iterations at each time step, these conditions can be enforced with high accuracy.
When these conditions are met exactly, we define the scheme as \textit{strongly} coupled.  

The scope of FSI methods can formally be divided into \textit{interface-tracking} and \textit{interface-capturing } methods.\cite{Frei2016}. In the \textit{Interface-tracking} method, the mesh moves to accommodate for the movement of the structure as it deformes the spatial domain occupied by the fluid. As such the mesh itself "tracks" the fluid-structure interface as the domain undergoes deformation. Such an approach is feasable as it allows for better control of mesh resolution near the interface, which in turn yeilds better control of this critical area. Among the  \textit{Interface-tracking} methods, the \textit{arbitary Lagrangian-Eulerian} formulation is the most well-known approach \cite{Richter2010a}, \cite{Frei2016}. In this approach the structure is given in its natural \textit{Lagrangian coordinate system}, while the fluid is transformed into an artificial \textit{Lagrangian} coordinate system. From this approach tracking of the interface $\hat{\Gamma}$ is more trivial, as it is fixed on a \textit{reference system} and can be tracked by mappings defined in Chapter 2. \\ \\

In \textit{interface-capturing} methods one distinguish the fluid and solid domains by some phase variable over a fixed mesh. As such one captures the interface  $\hat{\Gamma}$ as it´s moving over the mesh element. This method is often emplyed in simulations of multiphase-flow. This ideas was extended in \cite{Dunne2006a} were the authors proposed to transform the Lagrangian formulated structure equations in an Eulerian formulation, solving the system of equations in a fully Eulerian formulation. 
This approach was considered in this thesis, but implementation of tracking the interface in time was proven unsuccessful. Therefore ALE approach was finally chosen for this thesis. As such both the \textit{fully Eulerian} and \textit{ALE} concepts, strengths and weaknesses will be introduced following sub-chapters. \\

 
\section{Fully Eulerian}
This method keeps the fluid in its \textit{Eulerian coordinates}, and such can be seen as the natural counterpart of the ALE method \cite{Wick2013}. First proposed by , \cite{Dunne2006}. One motivation of such and approach is the handling of large-deformation, as the transformation to eulerian coordinates are purely natural.

\section{Arbitary Lagrangian Eulerian formulation}
MEANTION ALE CAN  BEHAVE EITHER EULERIAN AND LAGRANGIAN
The ALE method was initially developed to combine the strengths of the \textit{Lagranngian} and \textit{Eulerian} coordinate systems. As pointed out in chapter 2, the \textit{Lagrangian} description is useful for tracking particles as they are act upon by forces. Hence its main contribution is the ability to track interfaces and materials with history dependent properties.
In the ALE method one choose to keep the structure in its \textit{Lagrangian coordinate system}, while transforming the fluid domain into an artificial coordinate system similar to the \textit{Lagrangian coordinate system}. It is however important to note that there is no natural displacement in the fluid domain, hence this domain has no directly physical meaning \cite{Richter2010a}, \cite{Donea2004}. 
 
With this in mind, we will derive these transformations with the help of a new arbitary fixed reference system \ha{W}, following the ideas and approaches found in \cite{Richter2016}. Further we denote its deformation gradient as $\hat{F}_w$ and its determinant $\ha{J}_w$. Following the ideas from chapter 2, we introduce the invertibale mapping $\ha{T}_w : \ha{W} \rightarrow V(t)$ , with the scalar $\ha{f}(\ha{x}_W, t) = f(x,t) $ and vector $\hat{w}(\ha{x}_W, t) = \mathbf{w}(x,t) $ counterparts.\\ 
For $\ha{V} = \ha{W}$, $\ha{W}$ simply denotes the familiar Lagrangian description.
In the case $\ha{V} \neq \ha{W}$, $\ha{W}$ as pointed out earlier have no direct physical meaning.  Hence it is important to notice that the physical velocity $\hat{v}$ and the velocity of arbitrary domain $\pder{\ha{W}_w}{t}$ doesn't necessary coincide. This observation is essential, as we will soon see. \\

We will first define the transformation of spatial and temporal derivatives from $V(t)$ to $\ha{W}$ found in \cite{Richter2016}\\

\begin{lem}
Transformation of scalar spatial derivatives \\
\textit{Let f be a scalar function such that} $f: V(t) \rightarrow \mathbb{R}$, \textit{then} 
\begin{align}
\nabla f = \hat{F}_W^{-T} \hat{\nabla}\ha{f}
\end{align} 
\end{lem}

\begin{lem}
Transformation of vector spatial derivatives \\
\textit{Let \textbf{w} be a vector field such that} $\mathbf{w}: V(t) \rightarrow \mathbb{R}^d$, \textit{then} 
\begin{align}
\nabla \mathbf{w} = \hat{\nabla}\hat{w} \hat{F}_W^{-1} 
\end{align} 
\end{lem}

\begin{lem}
Transformation of scalar temporal derivatives \\
\textit{Let f be a scalar function such that} $f: V(t) \rightarrow \mathbb{R}$, \textit{then} 
\begin{align}
\pder{f}{t} = \pder{\ha{f} }{t} - (\hat{F}_W^{-1} \pder{\ha{T}_W}{t} \cdot \hat{\nabla}) \ha{f}
\end{align} 
\end{lem}

In addition we need a consistent way to transform the induced stresses in the \textit{Eulerian} coordinate system to $\ha{W}$. Hence we introduce the \textit{Piloa transformation}, found in most introduction courses in structure mechanics (ORANGE BOOK).
\\
\begin{lem}
T \\
\textit{Let \textbf{w} be a vector field such that} $\mathbf{w}: V(t) \rightarrow \mathbb{R}^d$, \textit{then the Piola transformation of w is defined by} 
\begin{align}
\mathbf{w} = \ha{J}_W \hat{F}^{-1}_W \hat{w}
\end{align} 
\end{lem}

The Piola transformation can be further extended to transform tensors, see \cite{Richter2016}, Orange book. This results is essential as it allows us to transform surface forces induced by the \textit{Cauchy stress tensor} on our arbitrary coordinate system $\ha{W}$. Lemma 1.4 brings us to \textit{the first Piola Kirchhoff stress tensor} $\hat{P} = \ha{J}_W \hat{\sigma}\hat{F}_W^{-T}$, mentioned in chapter 2. 

We now have the necessary tools to transform the conservation principles introduced in the fluid problem in chapter 2. Recall the Navier-Stokes equation defined in the \textit{Eulerian coordinate system} V(t).
\begin{align*}
&\rho \pder{\mathbf{v}}{t} + \rho \mathbf{v} \cdot \nabla \mathbf{v} =
\nabla \cdot \sigma + \rho \mathbf{f} \\
&\nabla \cdot \mathbf{v} = 0
\end{align*}
Using our newly introduced transformations of derivatives we map the equation to the arbitrary reference system $\ha{W}$. We will first consider the transformation of the\textit{material derivative}. By 
\begin{align*}
\frac{d \mathbf{v}}{dt}(x,t) = \pder{\mathbf{v}}{t}(x,t) + \nabla \mathbf{v}(x,t) \cdot \pder{x}{t} \\
\frac{d \mathbf{v}}{dt}(x,t) = \pder{\mathbf{v}}{t}(x,t) + \nabla \mathbf{v}(x,t) \cdot \mathbf{v}
\end{align*}
Note $\pder{x}{t}$ is the velocity of particles and not the transformation velocity $\pder{\ha{T}_W}{t}$. By lemma(1.1, 1.2, 1.3) we have  

\begin{align*}
\frac{d \mathbf{v}}{dt}(x,t) = 
\pder{\hat{v}}{t}(x,t) - (\hat{F}_W^{-1}\pder{\ha{T}_W}{t} \cdot \hat{\nabla})\hat{v}
+ \hat{F}_W^{-T}\hat{\nabla}\hat{v} \cdot \hat{v} \\
\mathbf{v} \cdot \nabla \mathbf{v} = \nabla \mathbf{v} \mathbf{v} = 
\hat{\nabla}\hat{v}\hat{F}_W^{-1}\hat{v} = (\hat{F}_W^{-1}\hat{v} \cdot \hat{\nabla})\hat{v} \hspace{4mm} \textit{FINN KILDE}
\end{align*}

These results can be used to show that

\begin{align*}
\pder{\mathbf{v}}{t} + \mathbf{v} \cdot \nabla \mathbf{v} =
\pder{\hat{v}}{t} + (\hat{F}_W^{-1}(\hat{v} - \pder{\ha{T}_W}{t}) \cdot \hat{\nabla}) \hat{v}
\end{align*}

By applying \textit{the first Piola Kirchhoff stress tensor} directly we transform the surface stress by 

\begin{align*}
\nabla \cdot \sigma = \nabla \cdot (\ha{J}_W \hat{\sigma}\hat{F}_W^{-T})
\end{align*}
In general, $\sigma$ is presumed on the form of a Newtonian fluid.
However special care must be taken, as $\sigma \neq \hat{\sigma}$ due to spatial derivatives within the tensor. Hence 
\begin{align*}
\sigma = -p I + \mu_f(\nabla \mathbf{v} + (\nabla \mathbf{v})^T \\
\hat{\sigma} = -\ha{p} I + \mu_f(\hat{\nabla}\hat{v}\hat{F}_W^{-1} +\hat{F}_W^{-T}\hat{\nabla}\hat{v}^T )
\end{align*} 

For the conservation of continuum we apply the \textit{Piola Transformation} such that

\begin{align*}
\nabla \cdot \mathbf{v} = \nabla \cdot (\ha{J} \hat{F}_W^{-1} \hat{v})
\end{align*}

With the introduced mapping identities we have the necessary tools to derive a full fluid-structure interaction problem defined of a fixed domain. Since the structure already is defined in its natural Lagrangian coordinate system, no further derivations are needed for defining the total problem.

\begin{equat}
\textit{ALE problem on a fixed domain}
\begin{align}
\ha{J} \pder{\hat{v}}{t} + \ha{J} (\hat{F}_W^{-1}(\hat{v} - \pder{\ha{T}_W}{t}) \cdot \hat{\nabla}) \hat{v}
= \nabla \cdot (\ha{J}_W \hat{\sigma}\hat{F}_W^{-T}) + \rho_f \ha{J} \mathbf{f}_f
\hspace{4mm} \text{in} \hspace{2mm} \Omega_f \\
\nabla \cdot (\ha{J} \hat{F}_W^{-1} \hat{v}) \hspace{4mm} \text{in} \hspace{2mm} \Omega_f \\
\rho_s \pder{\hat{v}_s}{t} = \nabla \cdot \mathbf{F}\mathbf{S} + \rho_s \mathbf{f}_s
\hspace{4mm} \text{in} \hspace{2mm} \Omega_s \\
\pder{\hat{v}_s}{t} = \hat{u}_s \hspace{4mm} \text{in} \hspace{2mm} \Omega_s \\
\hat{v}_s = \hat{v}_f \hspace{4mm} \text{on} \hspace{2mm} \Gamma_i \\
\ha{J}_W \hat{\sigma}\hat{F}_W^{-T} \cdot \mathbf{n} = 
\mathbf{F}\mathbf{S} \cdot \mathbf{n}  \hspace{4mm} \text{on} \hspace{2mm} \Gamma_i 
\end{align}
\end{equat}


\subsection*{Fluid mesh movement}
In the ALE framwork one of the most limiting factors is the degeneration of the mesh due to large deformations. Even the most advanced ALE formulated schemes reaches a limit when only re-meshing is nesecarry \cite{Wall12006}. Consequently the choice of an appropriate mesh moving technique is essential to preserve a feasible mesh quality for the simulation of fluid flow. Let the total domain deformation $\ha{T}(\ha{x}, t)$ be divided into the solid and fluid deformation $T_s$, $T_f$, were the fluid deformation is mapped to the arbitrary fixed reference system $\ha{W}$ presented in the last subsection.  
Then the ALE map $T_f$ on the form 
\begin{align*}
\ha{T}_f(\ha{x}, t) = \hat{x} + \hat{u}_f(\hat{x}, t)
\end{align*}
is constructed such that $\hat{u}_f$ is an extension of the solid deformation $\hat{u}_s$ from the interface to the fluid domain. Several extentions have been proposed throuhout the litteratur, and for an overview the reader is refered to \cite{MM2016}, and the reference therein. The construction of such extensions often involves solving some auxiliary problem on a partial differential equation(PDE) form, mainly of second-order. The \textit{Laplacian} and \textit{pseudo-elasticity} extentions are examples, which will be considered in this thesis. These extensions are beneficial in terms of simplicity and computational efficiency, but comes with a cost of user mesh customization. One often want to ensure a desired mesh position and some regularity of mesh spacing on the boundary, but it is impossible for second order extensions to specify both \cite{Helenbrook2003}. Therefore the author of \cite{Helenbrook2003}, proposes a fourth-order PDE, the \textit{biharmonic} extensions, to improve the regularity of the mesh deformation. \\

\subsubsection*{Laplacian model}

The main motivation for a \textit{Laplacian} smoothing is due to its simplicity and due to its property of bounding the interior displacements to the boundary values. 

\begin{align*}
&- \hat{\nabla} \cdot (\alpha^q \hat{\nabla} \hat{u}) = 0 \\
&\hat{u}_f = \hat{u}_s \hspace{2mm} \text{on} \hspace{2mm}  \Gamma \\
&\hat{u}_f = 0 \hspace{2mm} \text{on} \hspace{2mm} \partial \hat{\Omega}_f / \Gamma 
\end{align*}

Most favourable, the largest mesh deformation occuring should be confined to the interal part of the mesh as it causes the least distortion \cite{Jasak2006}. Therefore the introduced diffusion parameter $\alpha$, often raised to some power q, is introduced to manipulate this behaviour. The form of this paramter is often problem specific,  as selective treatment of the elements may vary from different mesh deformation problems. A jacobian based method was introduced in \cite{Stein}. In \cite{Jasak2006}, the authors reviewed several distance based options, where $\alpha$ was some function of the distance to the closest moving boundary. This method was adopted in this thesis on the form

\begin{align*}
\alpha(x) = \frac{1}{x^q}  \hspace{4mm} q = -1
\end{align*}

However as pointed out by \cite{Hsu}, one of the main disadvantages of using the linear Laplace equation is that the equation solves the mesh deformation components independently of one another. Say one have deformation only in the x-coordinate direction, the interior mesh points will only be moved along this deformation. Such a behavior restricts the use to the Laplace equation of mesh extrapolation purposes.

\subsubsection*{Linear elastic model}
Considering a linear elastic model for mesh moving was first introduced in \cite{Tezduyar1992}.  
Both \cite{Dwight}
\begin{align*}
&\nabla \cdot \sigma = 0 \\
&\sigma = \lambda Tr(\epsilon(u)) I + 2 \mu \epsilon(u) \\
&\epsilon(u) = \frac{1}{2}(\nabla u + \nabla  u^T)
\end{align*}

Where Lamé constants $\lambda$ and $\nu$ are given as

\begin{align*}
\lambda = \frac{\nu E}{(1 + \nu)(1 - 2\nu)} \hspace{2mm} \mu = \frac{E}{2(1 + \nu)}
\end{align*}

One of the main motivations for introducing such a model is the manipulation of Young's modulus $E$, and the poisson´s ration $\nu$. Recall that Young's modulus is the measurement of the a materials stiffness, while the poission's ratio describe the materials stretching in the transverse direction under extension in the axial direction. Manipulating these parameters one can influence the mesh deformation,
however the choice of these parameters have proven not to be consistent,  and to be dependent of the given problem.  \\

In \cite{Wicka} the author proposed a negative possion ratio, which makes the model mimic an auxetic material. Such materials becomes thinner in the perpendicular direction when they are submitted to compression, and this property is feasible for mesh under deformation. 

One of the most common approach is to set $\nu$ as a constant in the range $\nu \in [0, 0.5)$ and let $E$ be the inverse of the distance of an interior node to the nearest boundary surface \cite{MM2016}. 
The authors of \cite{Biedron} used this property and also argued that the Young's modulus also could be chosen as the inversely proportional to the cell volume. They also pointed out that both approaches would give the desired result that the small cells around the solid surface would modeled rigid, moving with the surface of the solid as it undergoes deformation. On the other hand cells further away will deform to counter the effects close to the solid surface.

\subsubsection*{Biharmonic model}
Using a biharmonic mesh deformation model provides further freedom in terms of boundary conditions, and the reader is encoured to consult \cite{Helenbrook2003} for a deeper review. We will in combination with \cite{Wicka} present two main approaches the biharmonic model is defined as 
\begin{align*}
\hat{\nabla}^2 \ha{u} = 0 \hspace{4mm} \text{on} \hspace{2mm} \hat{\Omega}_f 
\end{align*}
By introducing a second variable on the form $\ha{w} = - \hat{\nabla} \ha{u}$, we get the following system defined by 
\begin{align*}
&\hat{w} = -\hat{\nabla}^2\hat{u} \\
&- \hat{\nabla} \hat{w} = 0
\end{align*}

This model is defined in a mixed formulation, and as such the prize for quality and control of mesh deformation comes with the cost of more computational demanding problem. 

For the boundary conditions two types has been proposed in \cite{Wicka}. Let 
$\hat{u}_f$ be decomposed by the components $\hat{u}_f = (\ha{u}_f^{(1)}. \ha{u}_f^{(2)})$. Then we have

\begin{align*}
&\textbf{Type 1} \hspace{4mm} \ha{u}_f^{(k)} = \pder{\ha{u}_f^{(k)}}{n} = 0 \hspace{4mm} \partial \hat{\Omega}_f / \Gamma \hspace{2mm} \text{for} \hspace{1mm} k = 1, 2 \\
&\textbf{Type 2} \hspace{4mm} \ha{u}_f^{(1)} = \pder{\ha{u}_f^{(1)}}{n} = 0 
\hspace{2mm} \text{and} \hspace{2mm} \ha{w}_f^{(1)} = \pder{\ha{w}_f^{(1)}}{n} = 0 \hspace{4mm} \text{on} \hspace{1mm} \hat{\Omega}_f^{in} \cup \hat{\Omega}_f^{out} \\ 
&\hspace{17mm}  \ha{u}_f^{(2)} = \pder{\ha{u}_f^{(2)}}{n} = 0 
\hspace{2mm} \text{and} \hspace{2mm} \ha{w}_f^{(2)} = \pder{\ha{w}_f^{(2)}}{n} = 0 \hspace{4mm} \text{on} \hspace{1mm}  \hat{\Omega}_f^{wall}
\end{align*}

With the first type of boundary condition the model can interpreted as the bending of a thin plate, clamped along its boundaries. The form of this problem has been known since 1811, and its derivation has been connected with names like  French scientists Lagrange, Sophie Germain, Navier and Poisson \cite{Meleshko1997}.  

The main motivation for second type of boundary condition is for a rectangular domain where the coordinate axes match the Cartesian coordinate system \cite{Wicka}. In such a configuration, the mesh movement is only constrained in the perpendicular direction of the fluid boundary, leading to mesh movement in the tangential direction. This special case reduces the effect of distortion of the cells.  

\newpage

\newpage
\section{Discretization of the FSI problem}
Say something general of FSI discretization.. FEM, FVM, ...
In this thesis, the finite element method will be used to discretize the coupled fluid-structure interaction problem. It is beyound of scope  of this thesis, to thorough dive into the analysis of the finite element method regarding fluid-structure interaction problems. Only the basics of the method, which is nesecarry in order to define a foundation for problem solving will be introduced. 

\subsection{Finite Element method}
Let the domain $\Omega(t) \subset \mathbb{R}^d \ (d = 1, 2, 3) $  be a time dependent domain discretized a by finite number of d-dimentional simplexes.  Each simplex is denoted as a finite element, and the union of these elements forms a mesh. Further, let the domain be divided by two time dependent subdomains $\Omega_f$ and $\Omega_s$, with the interface $\Gamma = \partial \Omega_f \cap \partial \Omega_s$. The initial configuration $\Omega(t), t = 0 $ is defined as $\hat{\Omega}$, defined in the same manner as the time-dependent domain. $\hat{\Omega}$ is  known as the \textit{reference configuration}, and hat symbol will refer any property or variable to this domain unless specified. The outer boundary is set by $\partial \hat{\Omega}$ , with $\partial \hat{\Omega}^D$ and $\partial \hat{\Omega}^N$ as the Dirichelt and Neumann boundaries respectively. \\ \\

The family of Lagrangian finite elements are chosen, with the function space notation,
\begin{align*}
\hat{V}_{\Omega} := H^1(\Omega) \hspace{4mm} 
\hat{V}_{\Omega}^0 := H_0^1(\Omega)  
\end{align*}
where $H^n$ is the Hilbert space of degree n. \\
Let Problem 2.1 denote the strong formulation. By the introduction of appropiate trial and test spaces of our variables of interest, the weak formulation can be deduced by multiplying the strong form with a test function and taking integration by parts over the domain.  This reduces the differential equation of interest down to a system of linear equations. (skriv bedre..) \\
The velocity variable is continous through the solid and fluid domain
\begin{align*}
\hat{V}_{\Omega, \gat{v}} := \gat{v} \in H_0^1(\Omega), \hspace{2mm} 
\gat{v}_f = \gat{v}_s \ \text{on} \ \hat{\Gamma}_i \\
\hat{V}_{\Omega, \gat{\psi}} := \gat{\psi}^u \in H_0^1(\Omega), \hspace{2mm} 
\gat{v}_f = \gat{v}_s \ \text{on} \ \hat{\Gamma}_i 
\end{align*}
For the deformation, and the artificial deformation in the fluid domain let
\begin{align*}
\hat{V}_{\Omega, \gat{v}} := \gat{u} \in H_0^1(\Omega), \hspace{2mm} 
\gat{u}_f = \gat{u}_s \ \text{on} \ \hat{\Gamma}_i \\
\hat{V}_{\Omega, \gat{\psi}} := \gat{\psi}^v \in H_0^1(\Omega), \hspace{2mm} 
\gat{\psi}_f^v = \gat{\psi}_s^v \ \text{on} \ \hat{\Gamma}_i 
\end{align*}

For simplification of notation the inner product is defined as
\begin{align*}
\int_{\Omega} \gat{v} \ \gat{\psi} \ dx = (\gat{v}, \ \gat{\psi})_{\Omega}
\end{align*}
 

\subsection{Variational Formulation}
With the primaries set, we can finally define the discretization of the monolithic coupled fluid-structure interaction problem. For full transparency, variation formulation of all previous suggested mesh motion models will be shown. For brevity, the laplacian and linear elastic model will be shorted such that 
\begin{align*}
&\hat{\sigma}_{\text{mesh}} = \alpha \nabla \mathbf{u} \hspace{2mm} \text{Laplace} \\
&\hat{\sigma}_{\text{mesh}} =  \lambda Tr(\epsilon(\mathbf{u})) I + 2 \mu \epsilon(\mathbf{u}) \hspace{2mm} \text{Linear Elasticity} 
\end{align*}
Further, only the biharmonic model for the first type of boundary condition will be introduced as the second boundary condition is on a similar form.
  By the concepts of the Finite-element method, the weak variation problem yields.

\begin{prob}
\textit{Coupled fluid structure interaction problem for laplace and elastic mesh moving model.
Find $\bat{u}_s, \bat{u}_f, \bat{v}_s, \bat{v}_f, \ha{p}_f $ such that}
\begin{align*}
\big(\ha{J} \pder{\bat{v}}{t}, \ \gat{\psi}^u \big)_{\hat{\Omega}_f} +
\femf{\ha{J} (\hat{F}_W^{-1}(\bat{v} - \pder{\ha{T}_W}{t}) \cdot \hat{\nabla}) \bat{v}}{\gat{\psi}^u}
+ \femi{\ha{J}_W \hat{\sigma}\hat{F}_W^{-T} \bat{n}_f}{\gat{\psi}^u} \\
- \femf{\ha{J}_W \hat{\sigma}\hat{F}_W^{-T}}{\hat{\nabla}\gat{\psi}^u} -
\femf{\rho_f \ha{J} \mathbf{f}_f}{{\gat{\psi}^u}} = 0 \\
\fems{\rho_s \pder{\bat{v}_s}{t}}{\gat{\psi}^u} + \femi{\bat{F}\bat{S}\bat{n}_f}{\gat{\psi}^u}
- \fems{\bat{F}\bat{S}}{\nabla \gat{\psi}^u} - \fems{\rho_s \bat{f}_s}{\gat{\psi}^u} = 0 \\
\fems{\pder{\bat{v}_s - \bat{u}_s}{t}}{\gat{\psi}^v}  = 0\\
\femf{\nabla \cdot (\ha{J} \hat{F}_W^{-1} \bat{v})}{\gat{\psi}^p} = 0 \\
\femf{\hat{\sigma}_{\text{mesh}}}{\hat{\nabla}\gat{\psi}^u} = 0
\end{align*} 
\end{prob}

\begin{prob}
\textit{Coupled fluid structure interaction problem for biharmonic mesh moving model.
Find $\bat{u}_s, \bat{u}_f, \bat{v}_s, \bat{v}_f, \ha{p}_f $ such that}
\begin{align*}
\big(\ha{J} \pder{\bat{v}}{t}, \ \gat{\psi}^u \big)_{\hat{\Omega}_f} +
\femf{\ha{J} (\hat{F}_W^{-1}(\bat{v} - \pder{\ha{T}_W}{t}) \cdot \hat{\nabla}) \bat{v}}
{\gat{\psi}^u}
+ \femi{\ha{J}_W \hat{\sigma}\hat{F}_W^{-T} \bat{n}_f}{\gat{\psi}^u} \\
- \femf{\ha{J}_W \hat{\sigma}\hat{F}_W^{-T}}{\hat{\nabla}\gat{\psi}^u} -
\femf{\rho_f \ha{J} \mathbf{f}_f}{{\gat{\psi}^u}} = 0 \\
\fems{\rho_s \pder{\bat{v}_s}{t}}{\gat{\psi}^u} + \femi{\bat{F}\bat{S}\bat{n}_f}{\gat{\psi}^u}
- \fems{\bat{F}\bat{S}}{\nabla \gat{\psi}^u} - \fems{\rho_s \bat{f}_s}{\gat{\psi}^u} = 0 \\
\fems{\pder{\bat{v}_s - \bat{u}_s}{t}}{\gat{\psi}^v}  = 0\\
\femf{\nabla \cdot (\ha{J} \hat{F}_W^{-1} \bat{v})}{\gat{\psi}^p} = 0 \\
\femf{\hat{\nabla}\bat{u}}{\hat{\nabla}\gat{\psi}^{\eta}} - 
\femf{\bat{w}}{\hat{\nabla}\gat{\psi}^u} = 0 \\
\femf{\hat{\nabla}\bat{w}}{\hat{\nabla}\gat{\psi}^{v}} = 0
\end{align*}
\text{for the first type of boundary conditions introduced. } 
\end{prob}

Both problems introduced must handle the \textit{kinematic} and \textit{dynamic} boundary conditions in a consistent way. By a continuous velocity field on the whole domain, the \textit{kinematic} condition is strongly enforces on the interface $\hat{\Gamma}_i$
The continuity of normal stresses on the interface are defined as
\begin{align*}
 \femf{\ha{J}_W \hat{\sigma}\hat{F}_W^{-T} \bat{n}_f}{\gat{\psi}^u} = 
  \fems{\bat{F}\bat{S} \bat{n}_s}{\gat{\psi}^u}
\end{align*}
This condition is weakly imposed by omitting the boundary integral from the variational formulation \cite{Wick}, and it becomes an implicit condition for the system. \\

