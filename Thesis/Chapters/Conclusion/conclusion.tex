\chapter{Conclusion and further research}

In this thesis, a monolithic fluid-structure interaction solver in an arbitrary Lagrangian Eulerian description have been presented. This inline with the original goal of this thesis. However, verification of code was not achieved, however successful validation through the benchmark presented in~\cite{Hron2006} indicates the code still represents the mathematical model correctly enough. Thus it opens up the possibility that the verification of code test it self might have been erroneous. \\

An investigation of long term temporal stability of the FSI benchmark showed the implicit Crank-Nicolson was not applicable for time steps  $\Delta t > 0.001$. To overcome the stability issues, a shifted Crank-Nicolson scheme was introduced, were long time temporal stability was obtained for $\Delta t \leq 0.01$. Software profiling motivated run-time optimizations of the Newton solver, where a combination of Jacobian reuse and lower-order polynomials to assemble the Jacobian matrix proved most beneficial in terms of computational efficiency.  \\

In order to further explore all aspects of the FSI problem, I have also compared three different mesh lifting operators, focusing on mesh regularity and computational efficiency. The Laplace lifting operator proved to be the most efficient method, while the biharmonic operator was the most rigorous, however at the cost of computational cost. The linear elastic model failed for all tests expect the FSI-1 problem, meaning it was only valid for small deformations. 

\newpage
\subsection*{Future research}
To pursue a successful verification of code is the primary future goal, which is necessary remove any doubt that the mathematical model is solved correctly. In addition, several extensions are possible. Several publications have showed the linear elastic lifiting operator applicable for a wide range of FSI problems. Therefore, further investigation is planned to investigate why the operator did not perform in my work.  \\

Hoping to extend into 3D simulations, an implementation of iterative Krylov methods by block preconditioning are necessary, to overcome the CPU demanding nature of the monolithic FSI formulation. A substantial amount of time during the work of this thesis have been put into trying to implement the partitioned algorithm presented in \cite{Fernandez2007}. In the end, I was not able to finalize this project, but future research will be spent on finding the last mistakes. A projection method would allow for a wider variety of fluid schemes, making the whole fluid equation linear. Thus, having the potential of further speeding up the solution process.
