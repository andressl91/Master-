\chapter{Conclusion and further research}
A monolithic fluid-structrue interaction solver have been developed using the arbitary Lagrangian Eulerian method, suitable for lamiar viscus flow and large structure deformation. Verification of code was not achieved, however successfull validation by the fluid-structure interaction benchmark \cite{Hron2006} indicates the code represents the mathematical model correctly. A comparison of mesh lifting operators have been conducted, focusing on mesh regularity and computational efficiency. The harmonic operator proved to be the most efficient method, while the biharmonic operator produced the best evolution of mesh cells at the cost of increased computational time. \\

 An investigation of long term temporal stability of the FSI benchmark showed the implicit Crank-Nicolson was not applicable for time steps $\Delta t > 0.001$. To overcome the stability issues, a shifted Crank-Nicolson scheme was introduced, were long time temporal stability was obtained for $\Delta t \leq 0.01$. Software profiling motivated run-time optimizations of the Newton solver, where a combination of Jacobian reuse and lower-order polynomials to assemble the Jacobian matrix proved most beneficial in terms of computational efficiency.  \\

\subsection*{Future research}
To pursue a successful verification of code is the primary future goal, which is necessary remove any doubt that the mathematical model is solved correctly. In addition, several extensions are possible. Hoping to extend into 3D simulations, an implementation of iterative Krylov methods are necessary. 
