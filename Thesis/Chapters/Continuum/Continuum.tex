\chapter{Continuum Mechanics}
When studying the dynamics of a mediums with fluid or structure properties under the influence of forces, we need in some sense a good description of how these forces act and alter the system itself.

Any medium on a microscopic scale is built up of a structure of atoms, meaning we can observe "empty spaces " between each atom or discontinuities in the medium. Discribing any phsycial phenomen on larger scales in such a way are tedious and most often out of bounds due to the high number of particles. Instead we consider the medium to be continously distributed throughout the entire reagion it occupies. Hence we want to study some phsyical properties of the complete volume and not down on atomic scale. 

We consider the medium with continuum properties. By a continuum we mean a volume $V(t) \subset \mathbb{R}^3$ 
consiting of particles, which we observe for some properties. One property of interest could be the veloctity $\textbf{v}(x,t)$ for some point $x \in V(t)$ in time $t \in (0, T]$, which would mean the average velocity of the particles occupying this point \textit{x} at time \textit{t}  
The intension of this chapter is not to give a 
thorough introduction of continuum mechanics, but rather present key concepts needed for the evaluation of fluid-strucutre interaction.  

\section{Coordinate system}
We assume that our medium is continiously distributed throughout its own volume, and we start our observation of this medium
at som time $t_0$. As this choice is arbitary, we often choose to observe a medium in a stress free initial state. We call this state $V(t_0)$ of the medium as the \textit{reference configuration}. We let $V(t)$ for 
$t \geq t_0$ denote the \textit{current configuration}. \\ \\
Central for the coordinate systems introduced in this chapter is the concept of \textit{material} and \textit{spatial} points. \textit{Material} points are simply the points defining the material, moving with it as it undergoes movement. \textit{Spatial} points on the other hand is the relative measure of movement of the \textit{material} points. (Godt nok ??). This concept will be further explained throughout the chapter.

\subsection{Lagrangian}
As some medium is act upon by forces, one of the main properties of interest is the deformation of the medium. Hence we want to know the relative position of some particle from its initial configuration. \\
Let \^{x} be a particle in the reference  $\ha{x} \in \ha{V}$. 
Further let x(\^x, t) be the new location of a particle \^x for some time t such that $x \in V(t)$. We assume that no two particles $\ha{x}_a, \ha{x}_b \in \ha{V}$ occupy the same location for some time $V(t)$.
Then the transformation $\ha{T}(\ha{x}, t) = x(\ha{x}, t)$ maps a particle \ha{x} from the \textit{reference configuration} $\ha{V}$ to the  \textit{current configuration} $V(t)$
Assuming that the path for some \^{x} is continuous in time, we can define the inverse mapping $\ha{T}^{-1}(x, t) = \ha{x}(x, t)$, which maps $x(\ha{x}, t)$ back to its initial location at time $t = t_0$. \\
These mappings lets us track each particle from some \textit{reference configuration} to some deformed state at time t. 
Such a description of tracking each particle $\ha{x} \in \ha{V}$ is often denoted the \textit{Lagrangian Framework} and is a natural choice of describing structure mechanics. 

We define the \textit{deformation} 
\begin{align}
\ha{T}(\ha{x}, t) = \hat{u}(\ha{x},t) = x(\ha{x},t) - \ha{x} 
\end{align}

and the \textit{deformation velocity}
\begin{align}
\pder{\ha{T}(\ha{x}, t)}{t} = \hat{v}(\ha{x},t) = d_t x(\ha{x},t) = d_t \hat{u}(\ha{x},t) 
\end{align}

When tracking each particle as it moves, the \textit{material} and \textit{spatial} points coincide

\subsection{Eulerian}
Considering a flow of fluid particles in a river, a \textit{Lagrangian} description of the particles would be tidious as the number of particles entring and leaving the domain quickly rise to a immense number. 
Instead consider defining a view-point $V$ fixed in time, and monitor every fluid particle passing the coordinate $x \in V(t)$ as time elapses. Such a description is defined as the \textit{Eulerian framework.} 
Therefore the Eulerian formulation is natural for describing fluid dynamics. \\
We can describe the particles occupying the \textit{current configuration} $V(t)$ for some time $t \geq t_0$ 
\begin{align*}
x = \ha{x} + \hat{u}(\ha{x}, t)	
\end{align*}
Since our domain is fixed we can define the deformation for a particle 
occupying position $x = x(\ha{x},t)$ as
\begin{align*}
\textbf{u}(x, t) = \hat{u}(\ha{x}, t) = x - \ha{x}	\\
\end{align*}
and its velocity
\begin{align*}
\textbf{v}(x,t) = \partial_t u(x,t) = \partial_t \hat{u}(\ha{x},t) = \hat{v}(\ha{x},t)
\end{align*}

It is important to mention that the we are not interested in which particle is occupying a certain point in our domain, but only its properties. As such the \textit{material} and \textit{spatial} points doesn't coincide in the \textit{Eulerian formulation}


\section{Deformation gradients}
When studying continuum mechanics we observe continious mediums as they are deformed over time. These deformations
results in relative changes of positions due to external and internal forces acting.. These relative changes of postition is called
\textit{strain}, and is the primary property that causes \textit{stress} within a medium of interest \cite{Richter2016}. We define stress as the internal forces that particles within a continuous material exert on each other. \\

The equations of mechanics can be derived with respect to either a deformed or undeformend configuration of our medium of interest. The choice of refering our equations to the current or reference configuration is indifferent from a theoretical point of view. In practice however this choice can have a severe impact on our strategy of solution methods and physical of modelling.   \cite{Wriggers2006}. We will therefore define the strain measures for both configurations of our medium.  

\begin{defn}
Deformation gradient. 
\begin{align}
\hat{F} = I + \hat{\nabla} \hat{u} 
\end{align} 
\end{defn}

Mind that deformation gradient of $\hat{u}$ is which respect to the reference configuration. 
From the assumption that no two particles $\ha{x}_a, \ha{x}_b \in \ha{V}$ occupy the same location for some time $V(t)$, the presented transformation must be linear. As a consequence from the invertible matrix theorem found in linear algebra, the linear operator \textbf{F} cannot be a singuar.  
We define the  \textit{determinant of the deformation gradient} as \textit{J}, which denotes the local change of volume of our domain. 

\begin{defn}
Determinant of the deformation gradient
\begin{align}
J = det(\hat{F}) = det( I + \hat{\nabla} \hat{u} ) \neq 0
\end{align} 
\end{defn}


By the assumption that the medium can't be selfpenetrated, we must limit  J to be greater than 0 \cite{Wriggers2006}

\section{Measures of Strain and Stress}
The equations describing forces on our domain can be derived in accordinance with the current or reference configuration. With this in mind, different measures of strain can be derived with respect to which configuration we are interested in. We will here by \cite{Richter2016} show the most common measures of strain. We will first introduce the right \textit{Cauchy-Green} tensor \textbf{C}, which is one of the most used strain measures \cite{Wriggers2006}. \\ Uttrykk 1.3 fra Godboka, LAG TEGNING \\ 

Let $\ha{x}, \ha{y} \in \ha{V}$ be two points in our referemce configuration and let $\ha{a} = \ha{y} - \ha{x}$ denote the
length of the line bewtween these two points. As our domain undergoes deformation let 
$x = \ha{x} + \hat{u}( \ha{x} ) $ and $x = \ha{y} + \hat{u}( \ha{y} )  $ be the position of our points in the current configuration, and let $a = y - x$ be our new line segment. By \cite{Richter2016} we have by first order Taylor expansion

\begin{align*}
&y - x = \ha{y} + \hat{u}(\ha{y}) - \ha{x} - \hat{u}(\ha{x}) = \
\ha{y} - \ha{x} + \hat{\nabla}\ha{u}(\ha{x}) (\ha{y} - \ha{x}) 
+ \mathcal{O}(|\ha{y} - \ha{x} |^2) \\
&\frac{y - x}{|\ha{y} - \ha{x}|} = [I + \hat{\nabla}\hat{u}(\ha{x} ]  
\frac{\ha{y} - \ha{x}}{|\ha{y} - \ha{x}|} + \mathcal{O}(|\ha{y} - \ha{x} |) 
\end{align*}

This detour from \cite{Richter2016}  we have that 
\begin{align*}
&a = y - x = \hat{F}(\ha{x})\ha{a} +  \mathcal{O}(|\ha{a} |^2) \\
&|a| = \sqrt{ (\hat{F}\ha{a},\hat{F}\ha{a})+ \mathcal{O} (|\ha{a}^3|)  } = 
 \sqrt{ (\ha{a}^T, \hat{F}^T\hat{F}\ha{a})} + \mathcal{O} (|\ha{a}^2|)  
\end{align*}

We let $\ha{C} = \ha{F}^T \ha{F}$ denote the right \textit{Cauchy-Green tensor}.
By observation the Cauchy-Green tensor is not zero at the reference configuration 
\begin{align*}
\ha{C} =  \ha{F}^T \ha{F} = (I + \hat{\nabla} \ha{u})^T (I + \hat{\nabla} \ha{u}) = 1
\end{align*}

Hence it is convenient to introduce a tensor which is zero at the reference configuration. We define the \textit{Grenn-Lagrange strain tensor}, which arises from the squard rate of change of the linesegment \ha{a} and \textit{a}. By using the definition of the Cauchy-Green tensor we have the relation
\begin{align*}
&\frac{1}{2}(|a|^2 + |\ha{a}|^2) = \frac{1}{2}(\ha{a}^T\hat{C}\ha{a}
 -\ha{a}^T \ha{a} ) + \mathcal{O}(|\ha{a}^3| = 
 \ha{a}^T \big(\frac{1}{2} (\hat{F}^T \hat{F} - I) \big) \ha{a} 
 + \mathcal{O}(\ha{a}^3) \\
&\hat{E} = \frac{1}{2}(\hat{C} - I)
\end{align*}

Both the \textit{right Cauchy-Green tensor} $\hat{C}$ and the \textit{Green-Lagrange} $\hat{E}$ are refered to the Lagrangian coordinate system, hence the \textit{reference configuration}. \\
Using similar arguments (see \cite{Richter2016}, compsda) Eulerian counterparts of the Lagrangian stress tensors can be derived.

The \textit{left Cauchy-Green} strain tensor 
\begin{align*}
\mathbf{b} = \ha{F} \ha{F}^T = 
\end{align*}
and the \textit{Euler-Almansi} strain tensor
\begin{align*}
\mathbf{e} = \frac{1}{2} (I - \hat{F}^{-1}\hat{F}^{-T}) = \hat{F}^{-1}\hat{E}\hat{F}^{T}
\end{align*}

It is important to note that strain itself is nothing else than the measurement of line segments under deformation. Therefore strain alone is purely an observation, and it is not dependent on the material of interest. However one expects that a material undergoing strain, will give  forces within the material due to neighboring material interacting with one another. Therefore one derive materialspecific models to describe how a certain material will react to a certain amount of strain.\\
These strain measures are used to define models for \textit{stress}, which is responsible for the deformation in materials (cite holzapfel). The dimention of stress is force per unit area.

\section{Governing Equations}
The fully Fluid-structure interaction problem is based on equations of balance laws, with auxiliary kinematic, dynamic and material relations. In this section, assumptions regarding these relations will be described briefly. A deeper review of the full FSI problem will be considered in the next chapter. 

\subsection{Fluid}
We will throughout this thesis consider in-compressible fluids described by Navier-Stokes equations. We define the fluid density as $\rho_f$ and fluid viscosity $\nu_f$ to be constant in time. Our phsyical unknowns
fluid velocity $v_f$ and pressure $p_f$ both live in the time-dependent fluid domain  $\hat{\Omega}_f(t)$, with an eulerian configuration. Together with the equations of momentum and continuum, the Navier-Stokes equation is defined as,

\begin{equat}
\textit{Navier-Stokes equation}
\begin{align}
&\rho \pder{\mathbf{v}_f}{t} + \rho \mathbf{v}_f \cdot \nabla \mathbf{v}_f =
\nabla \cdot \sigma + \rho \mathbf{f}_f \hspace{4mm} \text{in} \hspace{2mm} \Omega_f \\
&\nabla \cdot \mathbf{v}_f = 0 \hspace{4mm} \text{in} \hspace{2mm} \Omega_f 
\end{align} 
\end{equat}
where $\mathbf{f}_s$ is some body force. 
Assuming a newtonian fluid the \textit{Cauchy stress sensor} $\sigma$ takes the form \newline $\sigma = -p_f I + \mu_f (\nabla \mathbf{v}_f + (\nabla \mathbf{v}_f)^T$.

Additional appropriate boundary conditions are supplemented to the equation for a given problem. The first type of of boundary conditions are Dirichlet boundary conditions, 
\begin{align}
\mathbf{v}_f = \mathbf{v}_f^D 
\hspace{4mm} \text{on} \hspace{2mm} \Gamma_f^D \subset \partial \Omega_f 
\end{align}
The second type of boundary condition are Neumann boundary conditions
\begin{align}
\sigma_f \cdot \mathbf{n} = \mathbf{g} 
\hspace{4mm} \text{on} \hspace{2mm} \Gamma_f^N \subset \partial \Omega_f 
\end{align}

\subsection{The solid}
For the structure we use the Vernant-Kirchhoff(STVK) model of deformation of solids, defined in a Lagrangian coordinate system. The solid is often characterized by the Possion ratio and Young modulus. Lamè coefficients  $\lambda_s$ and $\mu_s$ are then given by the relation.
\begin{align*}
&E_y = \frac{ \mu_s ( \lambda_s + 2 \mu_s) }{ ( \lambda_s + \mu_s ) } 
\hspace{5mm} \nu_s = \frac{\lambda_s}{2(\lambda_s + \mu_s)} \\
&\lambda_s = \frac{\nu E_y}{(1 + \nu_s)(1 - 2\nu_s)} \hspace{4mm} \mu_s = \frac{E_y}{2(1 + \nu_s)} \\
\end{align*}
\\
Our phsyical unknowns
solid velocity $v_s$ and deformation $u_s$ is our physical unknowns defined in a  time-dependent solid domain  $\hat{\Omega}_s(t)$.
The balance of solid momentum is given by

\begin{equat}
\textit{Solid momentum}
\begin{align}
\rho_s \pder{\mathbf{v}_s}{t} = \nabla \cdot \mathbf{T} + \rho_s \mathbf{f}_s
\hspace{4mm} \text{in} \hspace{2mm} \Omega_s
\end{align}
\end{equat}



where $\mathbf{f}_s$ is some body force, and $\mathbf{T}$ is the first \textit{Piola-Kirchhoff} stress tensor. The relation 
\begin{align}
\mathbf{T} = \mathbf{F} \mathbf{S}
\end{align}
connects the first \textit{Piola-Kirchhoff} to the second \textit{Piola-Kirchhoff} tensor. The material is chosen to be isotropic, hence \textbf{S} is on the form
\begin{align*}
&\mathbf{S} = \lambda_s tr(\mathbf{E}) + 2 \mu_s \mathbf{E} \\
&E = \frac{1}{2}(\mathbf{F}^T \mathbf{F} - \mathbf{I})
\end{align*} 
Since the solid deformation is a quantity of interest a kinematic condition must be defined for the system of the form
\begin{align}
\pder{\mathbf{v}_s}{t} = \mathbf{u_s} \hspace{4mm} \text{in} \hspace{2mm} \Omega_s
\end{align} 
One might ask the motivation of such an approach as the Lagrangian system could let us define the problem 
\begin{align}
\rho_s \ppder{\mathbf{u}_s}{t} = \nabla \cdot \mathbf{T} + \rho_s \mathbf{f}_s
\hspace{4mm} \text{in} \hspace{2mm} \Omega_s
\end{align}
directly solving for the main quantity of interest namely deformation. However solving for $\mathbf{v}_s$ is more convenient, as it lets us handle constraints for the fluid-structure interaction problem easier. As for the fluid problem we define Dirichlet and Neumann boundary conditions on the form

\begin{align*}
\mathbf{v}_s = \mathbf{v}_s^D 
\hspace{4mm} \text{on} \hspace{2mm} \Gamma_s^D \subset \partial \Omega_s  \\
\sigma_s \cdot \mathbf{n} = \mathbf{g}  
\hspace{4mm} \text{on} \hspace{2mm} \Gamma_s^N \subset \partial \Omega_s 
\end{align*}



