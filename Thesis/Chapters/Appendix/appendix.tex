\begin{appendices}
\addtocontents{toc}{\protect\setcounter{tocdepth}{0}}
%  \chapter{First appendix}
%  \section{First section}
%  \section{Second section}
  %\chapter{Second appendix}
  %\section{First section}
  %\section{Second section}
%\end{appendices}
\chapter{The deformation gradient}
\label{appendix:defgrad}
%\section{The deformation gradient}
Deformation is a major property of interest when a continuum is influenced by external and internal forces. The deformation results in relative change of position of material particles, called \textit{strain}, is the primary property that causes and describe \textit{stress}. Strain is purely an observation, not dependent on the material of interest. However one expects that a material undergoing strain, will apply forces within due to neighboring material particles interacting with one another. Therefore, material specific models are derived to describe how a certain material will react to a certain amount of strain. Strain measures are used to define models for \textit{stress}, which is responsible for the deformation in materials \cite{Holzapfel2000}. Stress is defined as the internal forces that particles within a continuous material exert on each other, with dimension force per unit area. \\
The equations of continuum mechanics can be derived with respect to either a deformed or undeformed configuration. The choice of referring our equations to the current or reference configuration is indifferent from a theoretical point of view. Regardless of configuration, the \textit{deformation gradient} and \textit{determinant of the deformation gradient} are essential measurement in structure mechanics. 
By \cite{Richter2016}, both configurations are considered.
\subsubsection*{Reference configuration}
\begin{defn}
Let $\bat{u}$ be a differential deformation field in the \textit{reference} configuration, $I$ be the Identity matrix and
the gradient $\hat{\nabla} = (\frac{\partial}{\partial x}, \frac{\partial}{\partial y}, \frac{\partial}{\partial z}) $. Then the \textit{deformation gradient} is given by,
\begin{align}
\bat{F} = I + \hat{\nabla} \bat{u} 
\end{align} 
\textit{expressing the local change of relative position under deformation.}
\end{defn}
\begin{defn}
Let $\bat{u}$ be a differential deformation field in the \textit{reference} configuration, $I$ be the Identity matrix and
the gradient $\hat{\nabla} = (\frac{\partial}{\partial x}, \frac{\partial}{\partial y}, \frac{\partial}{\partial z}) $. Then the \textit{determinant of the deformation gradient} is given by,
\begin{align}
J = det(\bat{F}) = det( I + \hat{\nabla} \bat{u} )
\end{align} 
\textit{expressing the local change of volume of the configuration.}
\end{defn}
\subsubsection*{Current configuration}
\begin{defn}
Let $\mathbf{u}$ be a differential deformation field in the \textit{reference} configuration, $I$ be the Identity matrix and
the gradient $\mathbf{\nabla} = (\frac{\partial}{\partial x}, \frac{\partial}{\partial y}, \frac{\partial}{\partial z}) $. Then the \textit{deformation gradient} is given by,
\begin{align}
\mathbf{F} = I - \mathbf{\nabla} \mathbf{u} 
\end{align} 
\end{defn}
\begin{defn}
Let $\mathbf{u}$ be a differential deformation field in the \textit{reference} configuration, $I$ be the Identity matrix and
the gradient $\mathbf{\nabla} = (\frac{\partial}{\partial x}, \frac{\partial}{\partial y}, \frac{\partial}{\partial z}) $. Then the \textit{determinant of the deformation gradient} is given by,
\begin{align}
J = det(\mathbf{F}) = det( I - \mathbf{\nabla} \mathbf{u} )
\end{align} 
\end{defn}
\end{appendices}
