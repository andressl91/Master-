\section{Measures of Strain and Stress}
The equations describing forces on our domain can be derived in accordinance with the current or reference configuration. With this in mind, different measures of strain can be derived with respect to which configuration we are interested in. We will here by \cite{Richter2016} show the most common measures of strain. We will first introduce the right \textit{Cauchy-Green} tensor \textbf{C}, which is one of the most used strain measures \cite{Wriggers2006}. \\ Uttrykk 1.3 fra Godboka, LAG TEGNING \\ 

Let $\ha{x}, \ha{y} \in \ha{V}$ be two points in our referemce configuration and let $\ha{a} = \ha{y} - \ha{x}$ denote the
length of the line bewtween these two points. As our domain undergoes deformation let 
$x = \ha{x} + \hat{u}( \ha{x} ) $ and $x = \ha{y} + \hat{u}( \ha{y} )  $ be the position of our points in the current configuration, and let $a = y - x$ be our new line segment. By \cite{Richter2016} we have by first order Taylor expansion

\begin{align*}
&y - x = \ha{y} + \hat{u}(\ha{y}) - \ha{x} - \hat{u}(\ha{x}) = \
\ha{y} - \ha{x} + \hat{\nabla}\ha{u}(\ha{x}) (\ha{y} - \ha{x}) 
+ \mathcal{O}(|\ha{y} - \ha{x} |^2) \\
&\frac{y - x}{|\ha{y} - \ha{x}|} = [I + \hat{\nabla}\hat{u}(\ha{x} ]  
\frac{\ha{y} - \ha{x}}{|\ha{y} - \ha{x}|} + \mathcal{O}(|\ha{y} - \ha{x} |) 
\end{align*}

This detour from \cite{Richter2016}  we have that 
\begin{align*}
&a = y - x = \hat{F}(\ha{x})\ha{a} +  \mathcal{O}(|\ha{a} |^2) \\
&|a| = \sqrt{ (\hat{F}\ha{a},\hat{F}\ha{a})+ \mathcal{O} (|\ha{a}^3|)  } = 
 \sqrt{ (\ha{a}^T, \hat{F}^T\hat{F}\ha{a})} + \mathcal{O} (|\ha{a}^2|)  
\end{align*}

We let $\ha{C} = \ha{F}^T \ha{F}$ denote the right \textit{Cauchy-Green tensor}.
By observation the Cauchy-Green tensor is not zero at the reference configuration 
\begin{align*}
\ha{C} =  \ha{F}^T \ha{F} = (I + \hat{\nabla} \ha{u})^T (I + \hat{\nabla} \ha{u}) = 1
\end{align*}

Hence it is convenient to introduce a tensor which is zero at the reference configuration. We define the \textit{Grenn-Lagrange strain tensor}, which arises from the squard rate of change of the linesegment \ha{a} and \textit{a}. By using the definition of the Cauchy-Green tensor we have the relation
\begin{align*}
&\frac{1}{2}(|a|^2 + |\ha{a}|^2) = \frac{1}{2}(\ha{a}^T\hat{C}\ha{a}
 -\ha{a}^T \ha{a} ) + \mathcal{O}(|\ha{a}^3| = 
 \ha{a}^T \big(\frac{1}{2} (\hat{F}^T \hat{F} - I) \big) \ha{a} 
 + \mathcal{O}(\ha{a}^3) \\
&\hat{E} = \frac{1}{2}(\hat{C} - I)
\end{align*}

Both the \textit{right Cauchy-Green tensor} $\hat{C}$ and the \textit{Green-Lagrange} $\hat{E}$ are refered to the Lagrangian coordinate system, hence the \textit{reference configuration}. \\
Using similar arguments (see \cite{Richter2016}, compsda) Eulerian counterparts of the Lagrangian stress tensors can be derived.

The \textit{left Cauchy-Green} strain tensor 
\begin{align*}
\mathbf{b} = \ha{F} \ha{F}^T = 
\end{align*}
and the \textit{Euler-Almansi} strain tensor
\begin{align*}
\mathbf{e} = \frac{1}{2} (I - \hat{F}^{-1}\hat{F}^{-T}) = \hat{F}^{-1}\hat{E}\hat{F}^{T}
\end{align*}