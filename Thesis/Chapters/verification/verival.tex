\chapter{Verification and Validation}
During the last decade, the amount of reaserch regarding simulations of physical problems has grown vast. Even though computers have changed our ways of solving real world problems, thrusting blindly numbers generated from a computer code has proven to be naive. It doesn't take a lot of coding experience before one realizes the many things that can brake down and produce unwanted and even suprisingly unexpected results. 
With this in mind, computer scientists and engineers need some common ground to check if a computer code works as expected. And it is here the framework of verification and validation plays and important role. \\

 An elegant and simple definition found throughout the litterature of verification and validation framwork, used  by Roache \cite{Roache}, states \textit{verification} as ``solving the equations right'', and  \textit{validiation} as ``solving the right equations". ``Solving the right equations"is rather vaguely, a measurement is needed. We will in this thesis use the more detailed  description found in \cite{Roache}.

\begin{quote}
The code author defines precisely what continuum partial differential equations and continuum boundary conditions are being solved, and convincingly demonstrates that they are solved correctly, i.e., usually with some order of accuracy, and always consistently, so that as some measure of discretization (e.g. the mesh increments) $\nabla \rightarrow 0$, the code produces a solution to the continuum equations; this is Verification.
\begin{flushright}
\textit{--- Roache, P.J.}
\end{flushright}
\end{quote}
 

Roach \cite{Roache2002},  further distinguish between the verification of \textit{code} and \textit{calculation}. Verification of code is seen as achieving the expected order of accuracy of the implementation, while verification of calculation is the measure of error against a known solution.

The goal of this chapter is to verify our implementations using the method of manufactured solution  (MMS).

\section{Verification of Code}
For scientists exploring physical phenomena,systems of partial differential equations (PDE´s) are often encountered. For their application it is important that these equations are implemented and solved numerically the right way.  Therefore insurence of right implemention is crucial. \\

Let a  partial differential equation of interest be on the form
\begin{align*}
\textbf{L}(\textbf{u}) = \textbf{f}
\end{align*}
\newpage
Here \textbf{L} is a differential operator, \textbf{u} is variable the of interest, and \textbf{f} is some sourceterm.

In the method of manufactured solution, one first manufactures a \textbf{u}, which is differentiated with \textbf{L} which yields a sourceterm  \textbf{f}. \\ The beauty of such an approach as mentioned by Roache \cite{Roache2002}, is that our exact solution can be constructed with physical reasoning. As such, code verificaion is purly a mathematical exercise where we are only interested if we are solving our equation right. These sentral ideas have existed for some time, but the  concept of combining manufactured exact solution in in partnership with refinement studies of of computational mesh was absent.  One of the earliest was Steinberg and Roache \cite{Steinberg1985} using these principles deliberately to estimate order of convergence\\

Throughout the literature there exist several approaches of code verification, but in the resent years verification using the Method of Manufactured Solutions has been favoured. 
To deeply verify the robustness of the method of manufactured solution,  a report regarding code verification using this approach was published by Salari and Knupp \cite{Biggs}. This thorough work applied the method for both compressible and incompressible time-dependent Navier-Stokes equation. To prove its robustness the authers  delibritary implemented  code errors in a Navier-Stokes solver presented in the report. In total 21 blind testcases where implemented, where different approaches of verifcation frameworks was tested. 
Of these 10 coding mistakes that reduces the observed order-of-accuracy was implemented. Here the method of manufactured solution captured all of them.

Even though the method of MMS a certain freedom, certain guidelines are presented in \cite{Biggs}. These are but not limited to

\begin{itemize}
\item To ensure theoretical order-of-accuracy, the manufactured solution should be constructed of polynomials, exponential or trigonometric functions to construct smooth solutions.
\item The solution should be utilized by every term in the PDE of interest, such that no term yields zero.
(få frem at en løsning må velges slik at ingen differentials blir 0
\item Certain degree to be able to calculate expected order of convergence (Få frem at må ha "grad nok" til å kunne regne convergencerate) 
\end{itemize}



                                                                                                                                                   
\section{Turek flag}