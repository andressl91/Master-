\documentclass[a4paper,norsk]{article}
\usepackage{preamble}

\begin{document}
\maketitle

\section*{Abstract}
Consider a tube filled with a newtonian viscous fluid initially at rest. Upon applying a pressuregradient, a
pressuredriven flow will occur known as Hagen–Poiseuille flow. Limiting ourselves to steady state case,
we will derive an analytical solution for this problem. Our goal is to validate the Oasis software,
a high-performance Navier-Stokes solver based on the opensoure Fenics software. The solver is based on the finite element method and
implemented in python.

\section*{Navier-Stokes Equation}
Let the velocity vector components be defined as $\textbf{v} = (v_r, v_\theta, v_z)$
The balance of momentum within the fluid introduces the famous Navier-Stokes eqations. By assuming
that fluid fluid is incompressible, we get the Navier-Stokes eqations as

\begin{align}
\frac{\partial \textbf{v}}{\partial t} + \textbf{v} \cdot \nabla \textbf{v} &=
-\frac{\nabla p}{\rho} + \frac{\mu}{\rho} \nabla^2 \textbf{v} \\
\nabla \cdot \textbf{v} &= 0
\end{align}

Limiting ourselves to the steady-state case of a fully developed fluid, following component equations.

\begin{align}
r &: \hspace{1cm} v_r\frac{\partial v_r}{\partial r} - \frac{v_\theta^2}{r} + v_z\frac{\partial v_r}{\partial z}
= -\frac{1}{\rho} \frac{\partial p}{\partial r} + \frac{\mu}{\rho}( \frac{\partial^2 v_r}{\partial r^2} +
\frac{1}{r} \frac{\partial v_r}{\partial r} + \frac{\partial^2 v_r}{\partial z^2} - \frac{v_r}{r^2} ) \\
\theta &: \hspace{1cm} v_r\frac{\partial v_\theta}{\partial r} + v_z\frac{\partial v_\theta}{\partial z} +
\frac{1}{r}v_\theta v_r = \frac{\mu}{\rho} (\frac{\partial^2 v_\theta}{\partial r^2} +
\frac{1}{r} \frac{\partial v_\theta}{\partial r} + \frac{\partial^2 v_\theta}{\partial z^2} -
\frac{v_\theta}{r^2} ) \\
z &: \hspace{1cm} v_r \frac{\partial v_z}{\partial r} + \frac{\partial v_z}{\partial z} =
\frac{1}{\rho} \frac{\partial p}{\partial z} + \frac{\mu}{\rho} (\frac{\partial^2 v_z}{\partial r^2} +
\frac{1}{r} \frac{\partial v_z}{\partial r} + \frac{\partial^2 v_z}{\partial z^2}
)
\end{align}

nevn senere hva som kan fjernes i teori




\section*{Laminar flow}
We will further simplify the Navier-Stokes equations by specifying certain conditions for the fluid flow.
We will for simplicity look at a constructed test case for a straight tube with constant diameter, with a
fully developed laminar flow. From this we can conclude that we have a fluid flow with the property of constant axial velocity, and no radial
or angular velocity. Further we will assume that the fluid is driven by a pressure gradient parallel to the axial direction.
As a result we have the following conditions.
\begin{align}
\frac{\partial \textit{v}}{\partial t} = 0 \\
\textit{v} \cdot \nabla(\textit{v} ) = 0 \\
v_r = v_\theta = 0
\end{align}

Now lets imagine extracting a ring-shaped control volume from the fluid flow and assess the forces acting on this volume.
Let this control volume have thickness \textit{dr} and width \textit{dx}.
SETT INN BILDE
Due to the balance of momentum the following forces must balance each other. As observed the only acting forces are the pressure and
viscous forces, denoted as \textit{p} and \textit{$\tau$}

\begin{align}
2\pi r P_x dr - 2\pi r P_{x+\delta x}dr + 2\pi r dx \tau_{r} - 2\pi r dx \tau_{r+\delta r} = 0 \hspace{2mm}* (\frac{1}{ 2\pi dr dx}) \\
r \frac{P_{x + \delta x} - P_x}{ dx} + \frac{r (\tau_{r + \delta r} - \tau_r)}{dr} = 0
\end{align}

In the limit dr, dx $\rightarrow$ 0, we get the following equation

\begin{align}
r \frac{d P}{d x} + \frac{ d(r \tau)}{d r} = 0
\end{align}

Now by replacing $\tau$ with $-\mu \frac{\d u}{d x}$ where the constant $\mu$ denotes the dynamic viscosity. Now, this means that
choice of x or r, the relation (10) must be fulfilled. Hence we must conclude that $\frac{d P}{d x}$ must be some constant. This constant
can be derived by changing our control volume with a slice at any point in the tube. Using the same relations it can be shown that

\begin{align}
\frac{d P}{d x} = \frac{-2 \tau_W}{R}
\end{align}
Where $\tau_W$ denotes the wall shear stress.

We now solve the second order differential equation(10), using double integration
\begin{align}
u(r) = \frac{1}{4 \mu} \frac{dP}{dx} + C_1 ln(r) + C_2
\end{align}
To remove the non-physical consequence of $\lim_{r \to 0} ln(r) \rightarrow \infty$, we choose $C_1 = 0$. By using our
second condition $u(R) = 0$ where R denotes the tube of the radius, the final analytical result yields

\begin{align}
u(r) = - \frac{1}{4 \mu}\frac{d P}{d x} \big(R^2 -  r^2 \big)
\end{align}

\section*{Setting up the problem}
Using Oasis, we can choose between two sets of solver methods, coupled and fracstep. \textbf{Coupled} is a steady-state
solver solving


For the computation we have to construct a mesh to do our calculations upon. In this problem we will use
Gmsh, a free 3D finite element grid generator. The problem will be solved on a tube with radius = 1 and length
z = 3.


\section{Computation}



\end{document}
